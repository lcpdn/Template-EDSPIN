%à placer dans le dossier Couverture-these

%Figure Tikz pour tracer les figures de la couverture
% tikzpicture to draw the cover


%Il faut rajouter le package \usetikzlibrary{fadings}


 \begin{tikzpicture}
 	        % right triangle 
            \foreach \x in {1,...,35}{
              \draw[very thick, color = couleur-ecole-recto] (14.015,6.4+\x*4.128/35-0.025) -- (\paperwidth,4.428+\x*4.128/35-0.025);
            }
            \fill[white] (14,6.4) -- (14,8.556) -- (14, 10.8) -- (\paperwidth, 10.8) -- (\paperwidth, 8.556) -- cycle; 

	        % middle triangle
            \foreach \x in {1,...,35}{
              \draw[very thick, color = couleur-ecole-recto] (7.02,10.528 +\x*4.128/35-0.025) -- (14,8.556+\x*4.128/35-0.025);
            }
            \fill[white] (7,10.528) rectangle (14.02, 17);
            \fill[white] (13.995,8.56) rectangle (14.1, 13);
            \shade[shading=axis, top color = white, path fading = south] (7,8.556) rectangle (14,10.54);
            
          % filled hexagon
            \fill[couleur-ecole-recto, fill opacity = 1] (-1,0) -- (-1,12.7817142857) -- (14,8.556) -- (14,6.4) -- (\paperwidth,4.4281) -- (\paperwidth,0) -- cycle;
  		
		      %points
            %à placer dans le dossier Couverture-these

\def\xyz{
16.3723200603856/6.07006877517398/0.979693182562524,
14.8348735902972/6.70352624286313/0.920837033525285,
15.9443459503929/5.85056750951652/0.513660117796151,
18.4795460226023/5.96108272900339/0.430315591494477,
20.5501761827624/7.1227714861746/0.113093701692133,
16.6647539132623/6.02450733517201/0.990210346755907,
17.2474683033424/9.08128348998245/0.206595522736581,
15.2425900889936/6.1777578816121/0.501843717637609,
14.8739429725302/5.50100499541354/0.700050079036327,
20.2676202915874/7.28330894630322/0.87765148465525,
18.9529343428413/5.45769727661441/0.169338966500959,
18.5124967217394/6.7214053849385/0.274517168677197,
14.259526803678/9.18227634143053/0.676369026345516,
18.1606758544879/6.80188449389315/0.878839613760174,
20.1945258436194/6.465000555334/0.165729234105952,
15.2327008461555/8.06544334876867/0.891951696064446,
17.2329827140654/4.41236583748225/0.0784237035038925,
18.9826381898775/5.43232654417096/0.416053761159688,
18.5692453738867/8.34045100602275/0.495148460559718,
15.5502435156335/8.70735728905601/0.38720412376666,
15.8496351701258/5.46549865953977/0.498771661037355,
17.2008472690505/7.75833609234053/0.63047478710939,
20.4266697553595/5.76648582114067/0.0556254906331534,
20.2721846698505/9.00451759950316/0.968424880978056,
15.2591851733785/5.65407202359739/0.830746281135502,
16.467262268122/9.27659885169027/0.631479724902459,
19.8471739737047/6.00799127035162/0.240630996872107,
15.7375227845279/7.85586386429912/0.824508361662589,
19.3622671213713/8.36598981423808/0.428374722019619,
18.2873736216747/7.76183318412434/0.568851367548572,
19.8963844092492/5.01925525101876/0.199413314380516,
20.6068603658136/4.93675729550071/0.405255730088962,
14.7086841543532/8.64163921988866/0.353815597514222,
16.683543119394/9.34949384228852/0.251346318383719,
18.407938706282/5.39571782849772/0.186285618931167,
16.3841745012409/8.02037263362788/0.102409977169413,
20.7732964752937/5.99099801429542/0.607284423857897,
16.3902427850882/5.95127109274762/0.476871604655865,
15.7232366548401/6.915394344283/0.798990602169386,
17.4378550636398/5.05765981787491/0.411231425034958,
16.0905227690776/4.85716678563312/0.951586944274749,
17.321972627476/5.81556869445782/0.174589246220256,
15.9363871473395/4.70372164263336/0.275648555956709,
17.7854375544118/4.70461730351412/0.559038149373835,
16.1299469298141/8.23243014779176/0.618535354765565,
16.7456522004498/5.40463279761927/0.0616736980738003,
16.901779395762/9.047771951112/0.760953195776694,
16.6227110894314/8.31308601164128/0.489216265679798,
16.6052131939601/4.79560824665545/0.685180263008388,
14.9444591806827/4.02268281285195/0.325555500976923,
15.6339678048682/8.04142003347734/0.678132616918648,
15.3009887646235/5.09024808336299/0.125020589981295,
14.1505920851309/8.58787795020434/0.530115213326068,
18.6744381850685/8.93342642223082/0.797596631298345,
16.4532202170908/9.4943863757835/0.183312956138322,
14.7677776326441/6.12196066350149/0.388437004539394,
19.3805591552648/9.26032186972039/0.0765857045794995,
14.8450361896405/4.53347330391933/0.993668031518369,
19.9307892842637/7.15185632023973/0.838292022063622,
16.1873612882768/9.23746665674372/0.834455758003474,
20.4701486107177/8.40930581464213/0.809686698153819,
16.3032334741463/9.49578648942177/0.297739524647659,
14.5358044206962/4.05274549913059/0.324103780523495,
19.3314430018155/4.06302479387903/0.019658907272616,
17.8074176886359/5.68943581848011/0.379538682663202,
20.644586525697/4.52770219609368/0.122208325586387,
17.079826407936/4.66094100890948/0.23608964653049,
14.7266398287733/4.2713695536761/0.0530131391105448,
16.3628158358932/8.94837577567134/0.472731715391544,
18.5298014790227/6.89176573417523/0.957755237950517,
16.1088740774268/6.36906773232808/0.23196394002416,
14.5175218724269/8.77204003864504/0.438131213129485,
19.8608096874358/6.62994285467343/0.240710061489216,
15.6657743694764/6.72920526271391/0.0211035721439959,
15.4856350969583/8.5266804581081/0.467010146569047,
17.0537458120015/8.74808994075915/0.136239080036511,
17.7861489851875/5.37484563298144/0.444504037555068,
14.6015864840162/4.77037245796043/0.642865616986866,
17.2412898301223/8.61100608095441/0.166685375295693,
17.7943146687979/5.53048894515719/0.570627688608427,
20.7551992050295/9.00112135899292/0.24011825425836,
20.7424955581344/7.2393631170055/0.153949486453373,
19.5210341219665/5.62404467035075/0.133326965935499,
18.0345879318154/4.11310541374052/0.870879201062408,
20.2443137064151/6.12902071181767/0.591688457180777,
17.439345834181/7.91812107952026/0.218733584649103,
16.561107891379/5.69118918880493/0.818780853699415,
17.6950091965803/6.97735388585267/0.710641165214654,
19.8297016181346/7.32556328425026/0.323576222897231,
14.525951378276/8.12405158535182/0.227207098322208,
14.10960406534/6.40666188125332/0.253060380542625,
20.2138364190746/9.99751942714338/0.35641027441715,
17.4418533436579/9.1617271007412/0.227200814822399,
15.8118522970804/5.04693391360776/0.401682693301288,
20.3522817966785/7.65794291546219/0.309167104674815,
16.0453659670823/5.26929909874569/0.187628991747945,
18.2900895663799/5.29749205809246/0.882575328875343,
16.5652845774638/8.05274105292896/0.609653665822905,
17.9810338322262/4.89328243290976/0.0462444871643782,
20.9801049392993/7.5239426385547/0.716927391520222,
20.631321780653/8.96593685313024/0.688464576852127,
17.2418263301704/5.77311049223974/0.357165588185292,
17.9654487962259/4.82393465346215/0.965661850389979,
19.3507394866092/9.14564074561877/0.913312638676918,
16.77644713035/8.74556236495816/0.1891811055582,
15.2644574724243/7.22803089945368/0.397318436077054,
19.7111568774018/6.41358215405445/0.152901375062558,
15.0421625145765/9.06980538750865/0.354155662878398,
15.4150606752012/7.33264282386405/0.589724061154516,
18.2882973747895/4.1088050222262/0.698847537511892,
14.498297737287/4.76642017284472/0.100114480144958,
18.1261828391084/4.75825152537962/0.32141332732947,
14.3818204044465/4.74907157884297/0.916982850703311,
15.0248253807689/7.42292731228749/0.151745856974391,
15.5870104931351/6.93353512698861/0.928903152768727,
20.2076179318319/5.58688049975702/0.830538180741619,
20.2731033360558/8.40868312635533/0.674140646041498,
15.8390575100549/5.06466813715822/0.92779796365497,
18.1238541776055/8.61869877950778/0.752085203550437,
16.8460281468982/6.99467576180007/0.118352392909995,
14.0362910204433/6.36446541799773/0.240991463512942,
19.817129704278/8.83053975836247/0.570387899346338,
15.5089218737464/7.96509330036255/0.894364965557346,
16.3021585237704/9.0040302686974/0.284940003929334,
14.3092284998385/9.01742615951714/0.375381957217448,
17.6480697223315/6.66342803304277/0.735821112010572,
18.2492506151941/4.26857729682298/0.0190586729941462,
15.0528889070166/5.13112501657443/0.0236751514265305,
14.5827364415285/4.60538601623427/0.447397271815824,
17.6209184430467/8.82075996340019/0.0948057135797121,
14.5876347178153/8.65407525780599/0.693092766936771,
20.1302526330761/5.43423040266538/0.499551325207774,
15.4996173417307/6.2928139046288/0.226520044411729,
18.2957265278786/7.21940402227361/0.591429532575073,
17.2124287492838/4.54284177773198/0.443278156833526,
15.2581743333123/7.50951911348562/0.0997387636951231,
15.3621717003032/9.20098078420365/0.0687789270447382,
14.7551539689624/7.70189856604193/0.561419338970805,
20.1871073708146/7.06594223739767/0.218974142977026,
18.4786007317387/9.35817940907775/0.706378138492962,
15.7078547412439/5.05960857389912/0.134417393172095,
17.0811244315002/8.3954238712683/0.62272778487932,
14.0830675279308/7.98610042976678/0.522611042181681,
16.805088512495/8.5592063881117/0.895874367320009,
16.163264385124/5.86202013039913/0.500308785675196,
18.3046284750459/5.85145083154672/0.294311903710961,
18.7028082095767/5.73978576126293/0.812599868836488,
16.5023072567074/5.50419707341678/0.12637083533173,
19.0863881039459/4.60980492658426/0.936792798296253,
20.9607346587034/7.04987845790371/0.935849438801013,
16.0934099935785/9.36117426252115/0.112185764496375,
18.3652941931658/9.56676783932099/0.788179986374352,
17.0678937057364/5.15356134815299/0.438734074849531,
16.9704668967962/8.14616710005019/0.794272990989324,
19.1942514860028/6.56514609433468/0.125418944178535,
20.2207551179805/7.03834409747716/0.115187560310088,
17.3698571526273/9.75256038347563/0.763821693912122,
16.017764857587/7.06746793463445/0.750288966453821,
17.8104043007521/8.82157911764749/0.461879296481043,
14.4941520471196/4.30664097989834/0.394470021862646,
15.2608919169717/5.67876394417792/0.421285152060273,
18.7748080286987/5.55692490241289/0.654490832537468,
17.5202554128253/5.56732451235112/0.849036763870399,
20.3587040263261/8.45745234473909/0.274609113240282,
19.3674693836698/4.94679694992343/0.132406027224997,
18.5481040922601/7.04359097701113/0.67884314850064,
17.0848468123201/9.05340416417131/0.531358158076337,
18.4595626447378/6.07660813213549/0.738403882088393,
20.0107747947026/8.78413225165326/0.718556337982146,
18.1632740181405/6.219981302962/0.0311936611850803,
14.1192743543391/6.87239884123161/0.165704805457227,
17.3271497850542/9.25894013370499/0.827641196155955,
16.5590177409817/8.53026800860481/0.0423397183533766,
19.0559979741205/8.63649812936811/0.888698689037383,
17.3043657396168/5.71063044072662/0.898490425592845,
15.5060854951615/9.10032337256262/0.480202172674982,
14.6401828346495/5.8608878251592/0.934682029800279,
18.5387348803107/5.07833316569688/0.633945591773834,
14.6104876586609/7.81181460639177/0.251474835266435,
16.6830601458145/7.84513403674985/0.369677785774647,
18.5668256628427/8.02522632373876/0.557503067206831,
20.8600310944022/7.20377342080125/0.804309985048998,
20.4031077059383/9.74065543815467/0.722553664719496,
19.7561411519126/5.50772538102649/0.149600408720076,
18.3852050236993/7.31996163577424/0.513910037519961,
15.1313129845389/7.79573602230758/0.722469200280321,
20.8822828656765/7.57353320143002/0.63410481210713,
17.5049782492021/9.84479331600695/0.807949212772581,
14.6582896227012/7.90824591924378/0.242167289587929,
17.9461363521278/6.58374064487321/0.512596476786809,
20.6958029415662/7.94411583031565/0.854980425311624,
20.8268416769343/9.49149882571349/0.173389098284733,
15.7832954261822/5.72551486520214/0.0425926766473209,
16.0536750739163/9.59411836418765/0.227177350530615,
16.3188044866964/9.5590562952868/0.269887375818524,
18.9423998681064/6.30360288434894/0.0279447534632059,
20.2394573566622/9.10885792418356/0.419905604670527,
16.7174507643465/4.96175392010899/0.909078766023541,
16.6134204920197/6.85428073674223/0.385077665186452,
16.4399120506886/4.07569543444547/0.0398269943588243,
20.7277898856154/7.11837417172072/0.045170907515773,
18.3328925014211/6.42562768872913/0.0628655727760853,
19.7834077451333/4.84874010154445/0.897026432737821,
16.6405820402794/7.20429563678228/0.0355008746042421,
17.0307870298534/4.20752710163814/0.483185828688947,
20.4277019844568/9.50064548828426/0.987873641636674,
18.3545218904247/7.20332513403992/0.839462504432239,
17.5352882241046/9.41151273792512/0.837299915831813,
19.2876301684972/8.95079307691612/0.249665387460154,
16.7854722613188/6.04648865140389/0.503966159394644,
16.7791027614333/5.740163282786/0.230881222241061,
19.3842521133687/4.57209286713675/0.821899720197282,
14.2335355134821/8.84103305277958/0.586217883578085,
14.7771505429623/8.97292745115066/0.973345783939925,
17.5688039236874/9.91428066743544/0.982989458317116,
18.1631342440174/9.95998654793151/0.703148043286015,
15.2287264022784/7.59697524434447/0.010027514955259,
18.9012601893602/8.39123388649215/0.115522562980719,
14.492430232027/9.07246781237697/0.168906477906117,
19.4425400244451/5.79525940396511/0.344361421990175,
16.1428904797979/4.9545762455556/0.642112653048244,
17.4126107925001/9.44184331432352/0.303384308991431,
16.850599081419/8.62126396015161/0.647112539921562,
16.6552013619814/9.89446180014701/0.214330998449769,
20.3170989371247/6.48005979346494/0.549375995380412,
16.5586505804776/4.26878170736264/0.78560788762258,
18.7866474751641/4.48102183395853/0.760394316730215,
19.7457007904787/5.68916867419396/0.159061062018409,
14.0628429278892/4.81081287643499/0.619869266584996,
19.1030042989311/9.8937367444571/0.239605890280499,
16.2515064793351/9.27109728305131/0.673652494092,
14.5132049552548/9.23767560699906/0.603865775914055,
19.6993317711958/6.84011651716607/0.801757336552046,
20.3363429367909/8.6494153824439/0.662585597813545,
15.6338890377391/9.82123274389733/0.584886545789265,
14.0294687131766/8.03863492443605/0.761336028858568,
20.0966663990468/7.74201927818488/0.366856835681479,
20.8568164716116/7.77737338950946/0.409915627629077,
18.0284423347085/9.02416338962615/0.409015246208295,
14.6196835648154/7.41335684992457/0.107095072415976,
15.7787101870698/4.83114547181184/0.676303887639591,
19.9753818583311/7.26333941944423/0.676190986342557,
18.7928305193707/7.57104603978094/0.717616754887866,
18.7913195547018/4.51809033109622/0.114117536339338,
14.1554886024965/7.50607300025017/0.447386727228027,
18.0779089420742/5.09635340921555/0.102482328143158,
15.1239737065013/6.87713372382273/0.217724298550511,
14.6100387132797/4.89432239919546/0.0396055615635472,
19.5461511099522/9.20351943475218/0.38472117221136,
19.9074630524185/6.57570021903499/0.239981270008865,
18.5422221444247/4.41393111277127/0.701621668079514,
14.4424042039113/5.15786824397891/0.452439593844786,
17.1031523960622/9.68073552118357/0.344886259643626,
16.7717713579607/8.08608115359829/0.509633898607324,
14.1595730729858/7.22626969505121/0.260443458459567,
18.0594381937543/7.10899109827046/0.294835711932412,
19.1685217770829/9.87656857736293/0.465643349893974,
16.4518464498276/6.2064856681592/0.197519247588994,
15.8300910871328/8.97249997194805/0.771976208416023,
14.9135813376226/6.17824169328859/0.451619846009771,
20.8039243932649/7.82965781645469/0.699436849532013,
15.4495892179529/8.51484486220857/0.319332670865505,
17.7719020811821/9.9454842244627/0.209424674478523,
15.3741570560084/5.19789369693244/0.00052145577629914,
18.424429349701/8.00806766139929/0.672239617358805,
19.0118285538188/5.24253212409696/0.783529427434992,
20.2066755815548/6.40715673604919/0.479681034941624,
18.6387219851881/8.26580833748471/0.0867031225844468,
18.787697474683/9.91681176824387/0.288428765104765,
18.9042934970107/5.0457135354497/0.999436849532013,
18.8201412909318/5.0616490091087/0.947968103494162,
18.5065198215592/5.16189369691571/0.719332670865506,
18.6065198290162/5.12879438833783/0.74731121690022,
18.4078824126734/5.19260981683374/0.707161853183784,
18.109816373469/5.26770981678337/0.966706494867673,
18.009816373461/5.29977833732523/0.942851900589568,
16.6788241267348/4.82373252313878/0.321723395182745,
19.4053833161557/6.12863570836574/0.61088857759278,
17.4203632237873/7.05139676189968/0.487622560338325,
20.0248957512294/8.61133373896888/0.665076567758917,
19.8890796141555/9.66479065098366/0.948533841464032,
17.7320565758826/4.61040963768288/0.64731121690022,
15.3154033090391/9.83594321179182/0.466757763312435,
17.7512877324238/8.72411100894909/0.511681413757901,
15.8920651982919/4.79514540555577/0.438577973748585,
15.6251511331616/8.84162763469901/0.551310823029412,
15.4159380276506/9.35275683834228/0.275429991805147,
18.6607310817554/9.14637558475488/0.585400550420447,
17.6857531407045/4.63864124291849/0.772536106543066,
16.5210120625444/8.52617740346766/0.334603243904965,
14.264981637347/8.17105082304714/0.536426154859166,
20.742485604123/9.97491279250994/0.46896260407905,
20.8264475390627/6.57354175721661/0.507161853183784,
19.7142269344989/4.14137823205315/0.966706494867673,
17.1227714767351/9.88583777835037/0.642851900589568,
15.873286447126/9.53234797879896/0.132462686196172,
18.9559036391679/4.90607871644234/0.348325209461866,
17.7367782083139/5.38904184773364/0.0863294336373429,
16.8147802780696/9.59363198076585/0.791245161558866,
16.1338566237222/5.33372679574781/0.851980109062184,
18.8276613710978/8.25386669150998/0.631224705148792,
20.7268079355087/7.33491631088355/0.684228274077821,
20.5884647322876/9.96852293079835/0.0852775057933815,
18.7092470159173/9.84380874004535/0.324731554468812,
14.5491464739225/4.90036515162344/0.722996370577809,
15.676497959298/9.84028441743745/0.967742164073016,
16.8483838866324/4.53758448457562/0.802980024442015,
19.5295808774465/7.86559543604889/0.257613902109499,
15.0837657814529/7.75071774499326/0.12125547435246,
20.2398594126608/6.35832288521687/0.184119836887249,
20.432698355166/4.07100281771601/0.848405368276539,
15.4636494826589/7.008195470021/0.901592729125198,
18.0511496452653/9.6097539933678/0.211988208636484,
17.6258529007341/9.83566991023567/0.201279932654726,
14.9940216111917/8.80718687742555/0.719682846160867,
18.6029679735522/4.46583758825315/0.858049667616388,
17.5022609466665/9.10687688370805/0.448189546817396,
17.9869797540092/9.76040648999398/0.797009989976817,
16.953052863676/6.42132163346196/0.661101833903077,
19.5212045475157/4.66359363766038/0.845334348816826,
20.22108560576/9.76059014052998/0.244551445004826,
20.3699689611468/5.99464171810216/0.643369231941243,
16.6992461680123/8.66086502768123/0.488075185767856,
20.9773829717446/4.994787853606/0.598559496989984,
17.5382327161129/8.5017440875887/0.804465182357859,
14.8022792500157/4.81945017744189/0.0351746528085691,
15.8839820586088/8.79395323265829/0.634911113048319,
16.3638199852121/5.91195221929119/0.185960863296543,
20.3102554686343/7.58096871396481/0.443596115170291,
20.3143199886366/8.83912178689017/0.205169721252597,
14.4085276314017/4.76794286820205/0.802368850635331,
16.4320647471171/7.6256191375054/0.282407359917214,
15.6401841766749/7.72593907172/0.506110878169321,
15.7606285058773/9.48386004954103/0.112496814616403,
18.306977034464/9.20119377615715/0.0879877730418316,
19.8392233383153/7.14779057645553/0.0516466577267024,
19.0400897886468/9.94426673390045/0.957202621251811,
14.9630345277469/7.30698014445097/0.0783823429459989,
17.8719807142165/6.0135831105783/0.542210606595412,
18.082065213982/7.70706302316409/0.23398293940748,
15.6384396783769/9.36626411572652/0.0976501586473481,
15.9812165358226/5.07695008212883/0.484802087437615,
19.4401861678733/5.19887342654827/0.872310268087383,
20.903857033834/5.10513813708296/0.331436785733478,
18.1010572381195/4.34769360559746/0.193520561859986,
14.0451808411544/8.15632971845775/0.147498339872538,
15.3423601908993/5.74385772165396/0.380268993827013,
18.2192365314027/4.26659624630871/0.803858309058704,
20.7658714943665/5.44698096194169/0.987827802524711,
16.5891944679177/6.00305467923644/0.219454297959618,
18.0737861169431/4.19478482152323/0.458699288299582,
17.7994647991384/5.75864259384899/0.332747893727349,
16.8898609745327/6.33675833754162/0.599658561960541,
16.6915295294031/8.38399637449687/0.782176513278184,
18.5663033700591/5.20724574044667/0.717318881294775,
16.7837843473007/8.19394553697761/0.320035531924108,
17.2677956644498/4.59901357379384/0.985789234870369,
18.5673085723147/7.57333897497883/0.630237476231728,
16.2271308017976/5.46809656808804/0.252030471192013,
20.6493805573043/9.29079667702789/0.930049658893345,
15.4051197448526/7.2186005446034/0.840738836289563,
18.025306360543/8.91831944253284/0.141713149081903,
19.3935581445573/8.81959527480564/0.984501310309885,
19.7785314417777/7.45375977743349/0.625071801235249,
17.0903917994316/4.193585013325/0.950926702015272,
14.969688480445/7.22185420460319/0.882105877820519,
17.7447286466266/7.88780617820399/0.340067494326275,
18.7937623817788/9.99266799231286/0.539715037060623,
16.9175356578661/6.24466565434792/0.227621933283926,
14.0655956908841/8.23457425395347/0.657432508319141,
15.8289943269764/7.00798012935856/0.473228896787696,
15.5470867394565/4.8020233595333/0.379058203155644,
20.2003202493182/5.44265409529563/0.768123651190487,
19.6814516933813/4.63975773724373/0.0178288087547238,
14.9036373559784/6.67715302574492/0.774875888322773,
18.8077494336189/5.94146251587449/0.436710056059616,
20.1111278241428/5.66892915128106/0.849516533106682,
18.7369551983404/9.12152813937329/0.986601204267153,
19.2669656739748/7.3727741380086/0.424383699005087,
14.9523587217891/5.11227754980112/0.703799966284744,
15.5557409782028/5.89570246031986/0.912665578576046,
15.388490893006/9.56505843347527/0.327411788489923,
14.8612015459632/6.94357031821749/0.618042341600984,
15.9095135542319/4.9502018475591/0.0420322532847331,
16.0727642051436/7.88467914874529/0.875599216307415,
20.5752661918839/8.09430375773411/0.880357350591246,
18.1231523256912/8.04671279274558/0.755346292133226,
17.7174653465154/9.92049734208693/0.391777498634398,
14.5146749763219/5.43123144620788/0.472618180484458,
15.7058377437905/7.51430732577881/0.396743051771478,
18.3647066276993/5.4172126128109/0.374747583388952,
19.2632875540006/7.96236180283879/0.780707289446982,
16.1668383666375/5.52505082660393/0.943162941892976,
20.5008734133614/7.72331113384042/0.162258533087917,
16.0309785568404/4.31207968470905/0.330594561734224,
14.2205081385173/7.08394025534949/0.602859372515564,
14.9677941462107/6.59306606626702/0.0872445604924984,
17.029282225891/7.04956762548139/0.643402529345905,
16.8840771398718/5.30706762248303/0.847443606453713,
14.8832214008299/4.38879256168778/0.99684724995123,
15.4509017918337/5.66818205846676/0.91033810146046,
14.1812522714127/9.89111921328401/0.0265018015257299,
15.9845566356038/8.30318455772527/0.374267661764771,
19.2023375162335/5.97921044480968/0.810477011299997,
19.7744943221528/7.59910996138196/0.43167345821673,
14.5900359198911/5.63073024295539/0.896129575377441,
18.7530995203426/4.90749641439094/0.112093500831289,
16.5829072254157/4.97852290452156/0.754277137690784,
14.2124571704908/6.96112969625079/0.658321713863943,
19.4244963190705/7.30399789523953/0.103713241571896,
18.7721022655518/5.66432241376335/0.419183430560566,
19.2132384344779/9.0264919684763/0.400094453090816,
19.740023915802/9.89140209437036/0.534268621974034,
17.9078293695389/9.35972592088164/0.774539115820307,
15.5614925030513/4.42269056353666/0.785063651977481,
14.0550912560734/4.52940639981309/0.125389432738382,
17.4204437276928/4.21646355767317/0.823931443320296,
20.9085386870685/6.85298424717986/0.252995137865804,
17.6888302820895/4.90832081268337/0.438611183573318,
19.9861057441174/5.974844201002/0.662539947017491,
19.0547664171833/7.11788637310053/0.344493723653163,
16.1545577139307/7.07353221989149/0.145276365585154,
20.5672921491154/9.45960933326373/0.164574558898071,
16.22441839529/9.97334147980343/0.439049364418046,
17.1076437760473/9.1091140433816/0.00144599666562995,
17.6790368767319/9.26728799724375/0.642331103365477,
18.0651141004142/6.21668958413385/0.323756475858088,
16.1284835684372/4.98871106038849/0.115121319004498,
14.1106979641303/5.91056281221479/0.520728649009163,
16.1126433626564/5.55724246777817/0.694418133877124,
18.6195144369961/5.27832173932962/0.424982804317419,
20.2212748243358/5.66954435559301/0.384298181736593,
15.1513836478351/9.44327873381879/0.460436474636706,
16.5708713346874/6.71490261730552/0.630688905638353,
15.7819203289768/7.00213567730256/0.428333784654202,
15.6335973390006/4.66216690816287/0.660625238025199,
17.7357436197891/9.82847496581993/0.0552464630525538,
15.7016424374012/4.17768857276566/0.246408963738679,
19.8047484554267/8.33503554115101/0.143559012724475,
14.8496139955129/7.20371648855543/0.0445504856684165,
14.0157805068086/8.35336241762629/0.752088448745111,
17.5450402704176/8.24655031060775/0.497178543924846,
16.4953165266402/7.17351391753714/0.860499693104398,
20.703165978913/8.34932492168779/0.383412792874712,
16.224889283421/8.45201651793398/0.246610617909403,
17.4418573290981/6.98911222254801/0.578320521273833,
14.9630019207924/6.36555285409344/0.799691049853342,
16.9792777904007/4.30217515612064/0.359017034811556,
20.1849774575123/9.55681772238435/0.972686341925448,
15.2668475924402/7.45342037073418/0.457461169627765,
17.3055106811129/9.62833494755854/0.237694180164435,
14.1571510714595/4.36230727465851/0.242723423118198,
20.8037959801776/9.6891102331836/0.453956671187427,
14.7350142898994/5.64086365169507/0.984043720290444,
17.0539066523645/5.46195241326776/0.745058324165415,
20.224276207347/7.4480565633585/0.446833006544693,
20.194230130201/7.65720657028235/0.877567358588805
}

\foreach \x/\y/\z in \xyz {
    \fill[couleur-ecole-recto, opacity=\z] (\x,\y) circle (.5pt);
}

            \shade[shading=axis, top color = white, path fading = south] (14,8.556) rectangle (\paperwidth,10);
\end{tikzpicture}